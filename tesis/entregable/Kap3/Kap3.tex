\chapter{Aspectos generales de los Morteros.}

\section{Características.}

Como un primer acercamiento al estudio de materiales de construcción se elaboraron 5 lotes de morteros, cuyas composiciones son distintas. Inicialmente cada lote consta de 4 placas. Es importante mencionar que estos no cumplen ningún tipo de reglamentación oficial. En las simulaciones realizadas para transmisión se colocan 10 placas y para retrodispersión 13. El lote de morteros 5 se empezará a analizar en detalle en el capitulo 6, pues en este se hacen las comparaciones entre simulación y experimento. 

(AGREGAR FORMA DE LOS MORTEROS.)

En las siguientes tablas se muestran las dimensiones de las placas correspondientes a cada lote. Estos tienen la forma que se muestra en la figura \ref{Fig:Diagrama-Mortero}.


\begin{table}[H]
	\centering
	\begin{tabular}{|c|c|c|c|c|c|c|c|}
		\hline
		\multicolumn{8}{|c|}{Laminas de Morteros1}                                                                             \\ \hline
		\multirow{Lamina \#} & Masa {[}g{]} & H1 {[}cm{]} & H2 {[}cm{]} & L1 {[}cm{]} & L2 {[}cm{]} & W1 {[}cm{]} & W2 {[}cm{]} \\ \cline{2-8} 
		& (+/- 0.01)   & \multicolumn{6}{c|}{(+/- 0.001)}                                                  \\ \hline
		1                          & 136.28       & 9.666      & 9.745      &9.663        & 9.647      & 0.886       & 0.916       \\ \hline
		2                          & 142.01       & 9.684      & 9.714      & 9.636       & 9.681      & 0.858       & 0.959       \\ \hline
		3                          & 145.92       & 9.726      & 9.572      & 9.579       & 9.546      & 0.880       & 1.069       \\ \hline
		4                          & 138.57       & 9.736      & 9.704      & 9.581       & 9.585      & 0.934       & 1.016       \\ \hline
	\end{tabular}
	\caption{Medidas experimentales del lote morteros1.}
	\label{t:medidas-morteros1}
\end{table}


 \begin{table}[H]
 	\centering
 	\begin{tabular}{|c|c|c|c|c|c|c|c|}
 		\hline
 		\multicolumn{8}{|c|}{Laminas de Morteros2}                                                                             \\ \hline
 		\multirow{Lamina \#} & Masa {[}g{]} & H1 {[}cm{]} & H2 {[}cm{]} & L1 {[}cm{]} & L2 {[}cm{]} & W1 {[}cm{]} & W2 {[}cm{]} \\ \cline{2-8} 
 		& (+/- 0.01)   & \multicolumn{6}{c|}{(+/- 0.001)}                                                  \\ \hline
 		1                          & 149.35       & 9.760       & 9.706      &9.518         & 9.523      & 0.757       & 0.974       \\ \hline
 		2                          & 157.36       & 9.742       & 9.753      &9.564         & 9.624      & 0.955       & 0.991       \\ \hline
 		3                          & 151.53       & 9.742       & 9.722      &9.615         & 9.493      & 0.923       & 0.956       \\ \hline
 		4                          & 136.97       & 9.757       & 9.794      &9.577         & 9.528      & 0.871       & 0.921       \\ \hline
 	\end{tabular}
 	\caption{Medidas experimentales del lote morteros2.}
 	\label{t:medidas-morteros2}
 \end{table}

  \begin{table}[H]
 	\centering
 	\begin{tabular}{|c|c|c|c|c|c|c|c|}
 		\hline
 		\multicolumn{8}{|c|}{Laminas de Morteros3}                                                                             \\ \hline
 		\multirow{Lamina \#} & Masa {[}g{]} & H1 {[}cm{]} & H2 {[}cm{]} & L1 {[}cm{]} & L2 {[}cm{]} & W1 {[}cm{]} & W2 {[}cm{]} \\ \cline{2-8} 
 		& (+/- 0.01)   & \multicolumn{6}{c|}{(+/- 0.001)}                                                  \\ \hline
 		1                          & 148.71       & 9.825        & 9.820      & 9.583         & 9.712      & 0.975       & 0.935       \\ \hline
 		2                          & 150.32       & 9.672        & 9.616      & 9.528         & 9.577      & 0.968       & 0.929       \\ \hline
 		3                          & 139.83       & 9.613        & 9.640      & 9.463         & 9.521      & 0.949       & 0.916       \\ \hline
 		4                          & 133.72       & 9.864        & 9.839      & 9.526         & 9.550      & 0.852       & 0.916       \\ \hline
 	\end{tabular}
 	\caption{Medidas experimentales del lote morteros3.}
 	\label{t:medidas-morteros3}
 \end{table}
 
 
    \begin{table}[H]
 	\centering
 	\begin{tabular}{|c|c|c|c|c|c|c|c|}
 		\hline
 		\multicolumn{8}{|c|}{Laminas de Morteros4}                                                                             \\ \hline
 		\multirow{Lamina \#} & Masa {[}g{]} & H1 {[}cm{]} & H2 {[}cm{]} & L1 {[}cm{]} & L2 {[}cm{]} & W1 {[}cm{]} & W2 {[}cm{]} \\ \cline{2-8} 
 		& (+/- 0.01)   & \multicolumn{6}{c|}{(+/- 0.001)}                                                  \\ \hline
 		1                          & 156.90       & 9.774         & 9.804      & 9.557          & 9.563      & 0.938       & 10.20       \\ \hline
 		2                          & 145.96       & 9.687         & 9.653      & 9.620          & 9.550      & 0.914       & 0.940       \\ \hline
 		3                          & 140.52       & 9.657         & 9.693      & 9.579          & 9.589      & 1.079       & 0.902       \\ \hline
 		4                          & 147.33       & 9.814         & 9.772      & 9.546          & 9.589      & 1.075       & 0.953       \\ \hline
 	\end{tabular}
 	\caption{Medidas experimentales del lote morteros4.}
 	\label{t:medidas-morteros4}
 \end{table}
 
 
     \begin{table}[H]
 	\centering
 	\begin{tabular}{|c|c|c|c|c|c|c|c|}
 		\hline
 		\multicolumn{8}{|c|}{Laminas de Morteros5}                                                                             \\ \hline
 		\multirow{Lamina \#} & Masa {[}g{]} & H1 {[}cm{]} & H2 {[}cm{]} & L1 {[}cm{]} & L2 {[}cm{]} & W1 {[}cm{]} & W2 {[}cm{]} \\ \cline{2-8} 
 		& (+/- 0.01)   & \multicolumn{6}{c|}{(+/- 0.001)}                                                  \\ \hline 		1                          & 136.58       & 9.762          & 9.734      & 96.67         & 96.10       & 1.009       & 0.877       \\ \hline
 		2                          & 133.28       & 9.660          & 9.646      & 96.00         & 96.42       & 0.850       & 0.887       \\ \hline
 		3                          & 128.12       & 9.629          & 9.673      & 95.11         & 95.30       & 0.927       & 0.824       \\ \hline
 		4                          & 130.35       & 9.670          & 9.677      & 95.77         & 96.93       & 0.869       & 0.912       \\ \hline
 	\end{tabular}
 	\caption{Medidas experimentales del lote morteros5.}
 	\label{t:medidas-morteros5}
 \end{table}
 
 Otro dato importante es la densidad del material, por ello se calcula de densidad de los morteros mediante la ecuación (\ref{densidad-mor1}).  Es importante mencionar que esta densidad es calculada para todas las placas por mortero, es decir, se tomó un lote y se le calculó el promedio de la masa y de las diferentes dimensiones, y con estos nuevos valores se hizo el cálculo. Este mismo proceso se repite para todos.

\begin{equation} \label{densidad-mor1}
    \begin{split}
	    \rho=\frac{masa}{volumen}
	\end{split}
\end{equation}

\begin{table}[H]
    \begin{center}
        \begin{tabular}{| c | c | }
            \hline
            Lote de Morteros & densidad (g/cm^3) \\ \hline
            1 & 1.62(4)  \\
            2 & 1.75(5) \\
            3 & 1.62(7) \\
            4 & 1.62(3) \\
            5 & 1.60(6) \\ \hline
        \end{tabular}
    \caption{Densidad de los diferentes Morteros.}
    \label{t:densi-morteros}
\end{center}
\end{table} 


También es necesario saber cuales son las composiciones y el porcentaje empleado en cada lote, así que en las tablas \ref{t:materiales-morteros1}, \ref{t:materiales-morteros2}, \ref{t:materiales-morteros3}, \ref{t:materiales-morteros4}, \ref{t:materiales-morteros5} muestras dichas composiciones. La composición para cada lote fue elegida de acuerdo a la utilidad que tiene en la práctica, es decir, se escogieron las mezclas mas usadas en las construcciones. Pues de acuerdo a lo que se desee construir se tiene una u otra mezcla. Todas las características proporcionadas son necesarias para construir las placas de morteros en Geant4.


\begin{multicols}{2}

\begin{table}[H]
	\centering
	\begin{tabular}{|c|c|c|}
		\hline
		\multirow{Material Mortero1} & \multicolumn{2}{c|}{4 placas} \\ \cline{2-3}
		& g         	& \%        	\\ \hline
		Cemento Portland      	& 320      	& 39.024    	\\ \hline
		Arena Peña         	& 300      	& 36.585    	\\ \hline
		Agua                  	& 200     	& 24.340     	\\ \hline
	\end{tabular}
	\caption{Proporción porcentual en la \\ 
	elaboración de morteros1.}
	\label{t:materiales-morteros1}
\end{table}


 \begin{table}[H]
 	\centering
 	\begin{tabular}{|c|c|c|}
 		\hline
 		\multirow{Material Mortero2} & \multicolumn{2}{c|}{4 placas} \\ \cline{2-3}
 		& g         	& \%        	\\ \hline
 		Cemento Portland      	& 160      	& 22.535    	\\ \hline
 		Arena Peña         	& 450      	& 63.380    	\\ \hline
 		Agua                  	& 100     	& 14.084     	\\ \hline
 	\end{tabular}
 	\caption{Proporción porcentual en la \\
 	elaboración de morteros2.}
 	\label{t:materiales-morteros2}
 \end{table}



  \begin{table}[H]
 	\centering
 	\begin{tabular}{|c|c|c|}
 		\hline
 		\multirow{Material Mortero3} & \multicolumn{2}{c|}{4 placas} \\ \cline{2-3}
 		& g         	& \%        	\\ \hline
 		Cemento Portland      	& 110      	& 15.492    	\\ \hline
 		Arena Sílice         	& 500      	& 70.422    	\\ \hline
 		Agua                  	& 100     	& 14.084     	\\ \hline
 	\end{tabular}
 	\caption{Proporción porcentual en la \\
 	elaboración de morteros2.}
 	\label{t:materiales-morteros3}
 \end{table}
 
 
 
 \begin{table}[H]
 	\centering
 	\begin{tabular}{|c|c|c|}
 		\hline
 		\multirow{Material Mortero4} & \multicolumn{2}{c|}{4 placas} \\ \cline{2-3}
 		& g         	& \%        	\\ \hline
 		Cemento Portland      	& 100      	& 10.989    	\\ \hline
 		Arena Sílice         	& 700      	& 76.923    	\\ \hline
 		Agua                  	& 110     	& 12.087     	\\ \hline
 	\end{tabular}
 	\caption{Proporción porcentual en la \\
 	elaboración de morteros4.}
 	\label{t:materiales-morteros4}
 \end{table}

 

\end{multicols}


 
 \begin{table}[H]
 	\centering
 	\begin{tabular}{|c|c|c|}
 		\hline
 		\multirow{Material Morteros5} & \multicolumn{2}{c|}{4 placas} \\ \cline{2-3}
 		& g         	& \%        	\\ \hline
 		Cemento Portland      	& 600      	& 75    	\\ \hline
 		Arena Sílice         	& 0      	& 0    	\\ \hline
 		Agua                  	& 200     	& 25     	\\ \hline
 	\end{tabular}
 	\caption{Proporción porcentual en la elaboración de morteros4.}
 	\label{t:materiales-morteros5}
 \end{table}

\section{Construcción.}

Como se puede apreciar la variación entre sus composiciones es notable, sin embargo, mas adelante veremos que las características a estudiar no varían significativamente entre los primeros 4 morteros, pues para el lote 5 las diferencias se hacen notables. Por esta razón se decide tomar estos últimos para hacer una comparación entre simulación y experimento. 

\vspace{5mm}

La construcción del lote de morteros 5 se hizo de la siguiente manera; se toman las proporciones mencionadas en la tabla (AGREGAR LA REFERENCIA) para el lote de morteros 5 y se vierten en un recipiente, posteriormente se mezclan. Una vez la mezcla es homogénea, es esparcida en unos moldes previamente hechos. Es importante mencionar que antes de hacer esto, a los moldes se les aplicó una capa de aceite (AGREGAR QUE TIPO DE ACEITE), esto con la idea de evitar que al endurecerse la mezcla fuese imposible retirar las placas sin ser dañadas.

\vspace{5mm}

Una vez hecho esto, se pusieron bolsas plásticas sobre el cemento preparado. De esta manera se mantiene la humedad, y así obtener placas con buena resistencia y mantener su integridad estructural intacta. En las construcciones, luego de hacer columnas, planchas, pisos y demás, es costumbre aplicar agua a estas con el fin de mantener la integridad de dichas estructuras. Por alguna razón que no entiendo, si no se hace esto empiezan a aparecer una serie de grietas. Esto se ve principalmente en los pisos, planchas y pañetes. 


\vspace{5mm}

Después de los procesos mencionados, se dejaron secar en el laboratorio por 3 días. Fueron 3 días porque no se tenía ingreso a la UN días antes. Es importante mencionar que al ser piezas pequeñas no requieren de tanto tiempo para secarse y endurecerse. Pasados estos días, se procedió a sacar las placas de los moldes. Se lograron extraer sin mayores complicaciones. 
Ya retiradas se midió el grosor y lados a cada una de ellas, Pues aunque los moldes tienen medidas bien definidas durante la elaboración y secado es posible que las dimensiones se alteren. En la siguiente tabla se muestran las dimensiones. Además el mas imágenes se muestra en proceso descrito anteriormente 

(FALTAN AGREGAR LAS FOTOS)