\chapter{Elaboración de los morteros.}

Como un primer acercamiento al estudio de los materiales de construcción se construyeron 5 lotes de morteros, cuyas composiciones se muestran a continuación. (AGREGAR LAS TABLAS QUE ESTÁN EN LA BITÁCORA.)

Como se puede apreciar la variación entre sus composiciones es notable, sin embargo, mas adelante veremos que las características a estudiar no varían significativamente entre los primeros 4 morteros, pues para el lote 5 las diferencias se hacen notables. Por esta razón se decide tomar estos últimos para hacer una comparación entre simulación y experimento. 

\vspace{5mm}

La construcción de estos morteros se hizo de la siguiente manera; se toman las proporciones mencionadas en la tabla (AGREGAR LA REFERENCIA) para el lote de morteros 5 y se vierten en un recipiente, posteriormente se mezclan. Una vez la mezcla es homogénea, es esparcida en unos moldes previamente hechos. Es importante mencionar que antes de hacer esto, a los moldes se les aplicó una capa de aceite (AGREGAR QUE TIPO DE ACEITE), esto con la idea de evitar que al endurecerse la mezcla fuese imposible retirar las placas sin ser dañadas.

\vspace{5mm}

Una vez hecho esto, se pusieron bolsas plásticas sobre el cemento preparado. De esta manera se mantiene la humedad, y así obtener placas con buena resistencia y mantener su integridad estructural intacta. En las construcciones, luego de hacer columnas, planchas, pisos y demás, es costumbre aplicar agua a estas con el fin de mantener la integridad de dichas estructuras. Por alguna razón que no entiendo, si no se hace esto empiezan a aparecer una serie de grietas. Esto se ve principalmente en los pisos, planchas y pañetes. 


\vspace{5mm}

Después de los procesos mencionados, se dejaron secar en el laboratorio por 3 días. Fueron 3 días porque no se tenía ingreso a la UN días antes. Es importante mencionar que al ser piezas pequeñas no requieren de tanto tiempo para secarse y endurecerse. Pasados estos días, se procedió a sacar las placas de los moldes. Se lograron extraer sin mayores complicaciones. 
Ya retiradas se midió el grosor y lados a cada una de ellas, Pues aunque los moldes tienen medidas bien definidas durante la elaboración y secado es posible que las dimensiones se alteren. En la siguiente tabla se muestran las dimensiones. Además el mas imágenes se muestra en proceso descrito anteriormente 

(FALTAN AGREGAR LAS FOTOS)